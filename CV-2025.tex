\documentclass[11pt,a4paper]{article}

% Packages
\usepackage[utf8]{inputenc}
\usepackage[T1]{fontenc}
\usepackage{geometry}
\usepackage{array}
\usepackage{longtable}
\usepackage{booktabs}
\usepackage{enumitem}
\usepackage{titlesec}
\usepackage{hyperref}
\usepackage{fancyhdr}
\usepackage{lastpage}

% Page geometry
\geometry{
    top=2.5cm,
    bottom=2.5cm,
    left=2.5cm,
    right=2.5cm
}

% Header/Footer
\pagestyle{fancy}
\fancyhf{}
\fancyfoot[C]{\thepage\ of \pageref{LastPage}}
\renewcommand{\headrulewidth}{0pt}

% Section formatting
\titleformat{\section}{\normalfont\large\bfseries}{\thesection.}{0.5em}{}
\titleformat{\subsection}{\normalfont\normalsize\bfseries}{}{0em}{}
\titlespacing*{\section}{0pt}{2ex plus 1ex minus .2ex}{1ex plus .2ex}

% Custom commands
\newcommand{\cvheader}[1]{\begin{center}\textbf{\underline{#1}}\end{center}}
\newcommand{\cvsubheader}[1]{\begin{center}\textit{#1}\end{center}}

% Table column types
\newcolumntype{L}[1]{>{\raggedright\arraybackslash}p{#1}}
\newcolumntype{C}[1]{>{\centering\arraybackslash}p{#1}}

\begin{document}

%==============================================================================
% TITLE
%==============================================================================
\cvheader{THE UNIVERSITY OF BRITISH COLUMBIA}
\cvsubheader{Curriculum Vitae for Faculty Members}

\vspace{1em}

\noindent\textbf{Date:} 3 April 2025 \hfill \textbf{Initials:}

\vspace{1.5em}

%==============================================================================
% SECTION 1-4: BASIC INFO
%==============================================================================
\noindent\textbf{1. SURNAME:} Rozali \hfill \textbf{FIRST NAME:} Moshe

\vspace{0.5em}
\noindent\textbf{2. DEPARTMENT/SCHOOL:} Physics

\vspace{0.5em}
\noindent\textbf{3. FACULTY:} Science

\vspace{0.5em}
\noindent\textbf{4. PRESENT RANK:} Professor \hfill \textbf{SINCE:} 2013

\vspace{1em}

%==============================================================================
% SECTION 5: POST-SECONDARY EDUCATION
%==============================================================================
\section*{\underline{5. POST-SECONDARY EDUCATION}}

\begin{tabular}{|L{4cm}|C{2cm}|L{4cm}|C{2.5cm}|}
\hline
\textbf{University or Institution} & \textbf{Degree} & \textbf{Subject Area} & \textbf{Dates} \\
\hline
Tel-Aviv University & BSc & Mathematics, Physics & 1987--1990 \\
\hline
University of Texas & PhD & Physics & 1991--1997 \\
\hline
\end{tabular}

\vspace{1em}

%==============================================================================
% SECTION 6: EMPLOYMENT RECORD
%==============================================================================
\section*{\underline{6. EMPLOYMENT RECORD}}

\textit{Prior to coming to UBC}

\vspace{0.5em}

\begin{tabular}{|L{5cm}|L{4cm}|C{3cm}|}
\hline
\textbf{University, Company or Organization} & \textbf{Rank or Title} & \textbf{Dates} \\
\hline
University of Illinois & Research Fellow & 1998 \\
\hline
Rutgers University & Research Fellow & 1999--2001 \\
\hline
\end{tabular}

\vspace{1em}

\textit{(b) At UBC}

\vspace{0.5em}

\begin{tabular}{|L{6cm}|C{6cm}|}
\hline
\textbf{Rank or Title} & \textbf{Dates} \\
\hline
Assistant Professor & 2001--2007 \\
\hline
Associate Professor & 2007--2013 \\
\hline
Professor & Since 2013 \\
\hline
\end{tabular}

\vspace{0.5em}

\noindent $\bullet$ Date of granting of tenure at U.B.C.: September 2007.

\vspace{1em}

%==============================================================================
% SECTION 7: LEAVES OF ABSENCE
%==============================================================================
\section*{\underline{7. LEAVES OF ABSENCE}}

\begin{tabular}{|L{5cm}|L{4cm}|L{4cm}|}
\hline
\textbf{University, Company or Organization at which Leave was taken} & \textbf{Type of Leave} & \textbf{Dates} \\
\hline
N/A & Parental & Jan--Apr, Sep--Dec 2014 \\
\hline
 & Sabbatical & Jan--June 2017 \\
\hline
\end{tabular}

\vspace{1em}

%==============================================================================
% SECTION 8: TEACHING
%==============================================================================
\section*{\underline{8. TEACHING}}

\textit{Courses Taught at UBC}

\vspace{0.5em}

\begin{longtable}{|L{2cm}|C{1.5cm}|C{1.2cm}|C{1.5cm}|C{1.5cm}|C{1.5cm}|C{1cm}|C{1cm}|}
\hline
\textbf{Session} & \textbf{Course Number} & \textbf{Hours} & \textbf{Class Size} & \textbf{Lecture Hours} & \textbf{Tutorials} & \textbf{Labs} & \textbf{Other} \\
\hline
\endfirsthead
\hline
\textbf{Session} & \textbf{Course Number} & \textbf{Hours} & \textbf{Class Size} & \textbf{Lecture Hours} & \textbf{Tutorials} & \textbf{Labs} & \textbf{Other} \\
\hline
\endhead
Spring 2002 & 508 & 39 & 7 & 39 & & & \\
\hline
Fall 2002 & 526 & 39 & 8 & 39 & & & \\
\hline
Spring 2003 & 401 & 39 & 9 & 39 & & & \\
\hline
Fall 2003 & 526 & 39 & 11 & 39 & & & \\
\hline
Spring 2004 & 401 & 39 & 21 & 39 & & & \\
\hline
Fall 2004 & 526 & 39 & 25 & 39 & & & \\
\hline
Summer 2005 & 250 & 39 & 60 & 39 & 12 & & \\
\hline
Fall 2005 & 504 & 39 & 20 & 39 & & & \\
\hline
Summer 2006 & 250 & 39 & 60 & 39 & 12 & & \\
\hline
Fall 2006 & 504 & 39 & 20 & 39 & & & \\
\hline
Spring 2007 & 529 & 39 & 20 & 39 & & & \\
\hline
Summer 2007 & 250 & 39 & 60 & 39 & 12 & & \\
\hline
Fall 2007 & 504 & 39 & 20 & 39 & & & \\
\hline
Spring 2008 & 402 & 39 & 30 & 39 & & & \\
\hline
Spring 2008 & 529b & 39 & 10 & 39 & & & \\
\hline
Fall 2008 & 526 & 39 & 22 & 39 & & & \\
\hline
Spring 2009 & 402 & 39 & 30 & 39 & & & \\
\hline
Fall 2009 & 526 & 39 & 25 & 39 & & & \\
\hline
Spring 2010 & 402 & 39 & 30 & 39 & & & \\
\hline
Fall 2010 & 407 & 39 & 36 & 39 & & & \\
\hline
Spring 2011 & 402 & 39 & 37 & 39 & & & \\
\hline
Fall 2011 & 403 & 39 & 35 & 39 & & & \\
\hline
Spring 2012 & 407 & 39 & 35 & 39 & & & \\
\hline
Fall 2012 & 403 & 39 & 30 & 39 & & & \\
\hline
Spring 2013 & 407 & 39 & 46 & 39 & & & \\
\hline
Fall 2013 & 407 & 39 & 48 & 39 & & & \\
\hline
Spring 2015 & 216 & 65 & 50 & 39 & 26 & & \\
\hline
Fall 2015 & 410 & 50 & & 39 & & & \\
\hline
Spring 2016 & 216 & 120 & & 39 & 26 & & \\
\hline
Fall 2016 & 410 & 50 & & 39 & & & \\
\hline
Fall 2017 & 410 & 70 & & 39 & & & \\
\hline
Spring 2018 & 529 & 9 & & 39 & & & \\
\hline
Fall 2018 & 410 & 60 & & 39 & & & \\
\hline
Spring 2019 & 508 & 6 & & 39 & & & \\
\hline
Fall 2019 & 170 & 160 & & 39 & & & \\
\hline
Spring 2020 & 508 & 14 & & 39 & & & \\
\hline
Fall 2020 & 170 & 220 & & 39 & & & \\
\hline
Spring 2021 & 508 & 12 & & 39 & & & \\
\hline
Fall 2021 & 170 & 200 & & 39 & & & \\
\hline
Spring 2022 & 508 & 6 & & 39 & & & \\
\hline
Fall 2022 & 407 & 42 & & 39 & & & \\
\hline
Spring 2023 & 529 & 12 & & 39 & & & \\
\hline
Fall 2023 & 407 & 40 & & 39 & & & \\
\hline
Spring 2024 & 508 & 10 & & 39 & & & \\
\hline
Fall 2024 & 407 & 40 & & 39 & & & \\
\hline
Spring 2025 & 203 & 140 & & 39 & & & \\
\hline
\end{longtable}

\vspace{0.5em}

\noindent(\textbf{Course names:} Physics 170 -- Mechanics, Phys 203 -- Thermal Physics, Phys 216 -- Mechanics, Phys 250 -- Modern Physics, Phys 401 -- Electricity and Magnetism, Phys 402 -- Applications of Quantum Mechanics, Phys 403 -- Statistical Mechanics, Phys 407 -- General Relativity, Phys 410 -- Computational Physics, Phys 504 -- Relativity and Electromagnetism, Physics 508 -- Quantum Field Theory, Phys 526 -- Quantum Electrodynamics, Phys 529 -- String Theory, Phys 529b -- Gravitational Aspects of String Theory)

\vspace{1em}

%------------------------------------------------------------------------------
% SECTION 8b: HIGHLY QUALIFIED PERSONNEL
%------------------------------------------------------------------------------
\subsection*{(b) \textit{Highly Qualified Personnel Supervised}}

\textbf{Graduate Students:}

\vspace{0.5em}

\begin{longtable}{|L{3.5cm}|C{2cm}|C{2cm}|C{2.5cm}|L{2.5cm}|}
\hline
\textbf{Student Name} & \textbf{Program Type} & \textbf{Starting Year} & \textbf{Finishing Year} & \textbf{Supervisory Role} \\
\hline
\endfirsthead
\hline
\textbf{Student Name} & \textbf{Program Type} & \textbf{Starting Year} & \textbf{Finishing Year} & \textbf{Supervisory Role} \\
\hline
\endhead
Philip DeBoer & PhD & 2001 & 2005 & Supervisor \\
\hline
Karene Chu & MSc & 2001 & 2004 & Supervisor \\
\hline
Jianyang He & MSc & 2003 & 2005 & Supervisor \\
\hline
Callum Quigley & MSc & 2003 & 2005 & Supervisor \\
\hline
Marc Lalancette & MSc & 2005 & 2008 & Supervisor \\
\hline
Lionel Brits & MSc & 2005 & 2007 & Supervisor \\
\hline
Jianyang He & PhD & 2005 & 2010 & Supervisor \\
\hline
Lionel Brits & PhD & 2007 & (did not complete) & Supervisor \\
\hline
Alex Rohvarger & MSc & 2010 & 2012 & Co-Supervisor \\
\hline
Darren Smyth & PhD & 2010 & 2016 & Supervisor \\
\hline
Anson Wong & MSc & 2012 & 2014 & Supervisor \\
\hline
Alexandre Vincart-Emard & PhD & 2012 & 2017 & Supervisor \\
\hline
Anson Wong & PhD & 2014 & (did not complete) & Supervisor \\
\hline
Wyatt Reeves & MSc & 2017 & 2018 & Supervisor \\
\hline
Sean Cooper & MSc & 2017 & (did not complete) & Supervisor \\
\hline
Wyatt Reeves & PhD & 2019 & 2023 & Supervisor \\
\hline
Sean Cooper & PhD & 2019 & (did not complete) & Supervisor \\
\hline
Kris Samant & MSc & 2023 & (did not complete) & Supervisor \\
\hline
Chuanxin Cui & MSc & 2024 & & Supervisor \\
\hline
\end{longtable}

\noindent(Additionally I am on the supervisory committee of about 6--8 PhD students at any given time).

\vspace{1em}

\textbf{Undergraduate Students:}
\begin{itemize}[noitemsep]
    \item Marc Lalancette (Summer 2003 -- Matrix Models)
    \item Danica Marsden (Fall 2004 -- String Cosmology)
    \item Sara Bartoloucci (Imperial College), Summer 2016
    \item Robert King (Imperial College), Summer 2019
    \item Phillip Bemont, Summer 2019, and 449 2019--2020
\end{itemize}

\vspace{1em}

\textbf{Postdoctoral Fellows:}

\noindent Our string theory group supported the following postdocs jointly:

\begin{itemize}[noitemsep]
    \item Kazuyuki Furuuchi (2001--2004)
    \item Ehud Schreiber (2002--2004)
    \item Dominic Brecher (2002--2005)
    \item Kazumi Okuyama (2004--2006)
    \item Paul Koerber (2004--2005)
    \item Pallab Basu (2007--2010)
    \item Anindya Mukherjee (2007--2010)
    \item Klaus Larjo (2008--2011)
    \item Joshua Davis (2009--2011)
    \item Tommy Levi (2009--2012)
    \item Bartek Czech (2009--2012)
    \item Nima Lashkari (2012--2015)
    \item Evgeny Sorkin (2012--2013)
    \item Omid Saremi (2013--2015)
    \item Josephine Suh (2015--2017)
    \item Jaehoon Lee (2015--2017)
    \item Benson Way (2016--2019)
    \item Eliot Hijano (2016--2019)
    \item Felix Haehl (2016--2019)
    \item Tarek Anous (2016--2019)
    \item Eric Mintun (2016--2019)
    \item Jason Pollack (2017--2020)
    \item James Sully (2018--2022)
    \item Lampros Lamprou (2020--2023)
    \item Arjun Kar (2020--2023)
    \item Charles Marteau (2020--2023)
    \item Aidan Chatwin-Davies (2021--2023)
    \item Felipe Rosso (2021--2023)
    \item Panagiotis Betzios (2021--2024)
    \item Jeremy van der Heijden (2024--)
    \item Sean McBride (2024--)
    \item Alejandro Vilar Lopez (2024--)
\end{itemize}

\vspace{1em}

\textit{Visiting Lecturer (indicate university/organization and dates)}

\noindent Long Term Visitor, Perimeter Institute for Theoretical Physics, January -- May 2005, January -- May 2006.

\noindent Visiting Scientist, Syracuse University, 2004--2006.

\vspace{1em}

\subsection*{(d) \textit{Other}}

\noindent Thesis Examiner, Perimeter Scholars International, 2011, 2013.

\vspace{1em}

%==============================================================================
% SECTION 9: SCHOLARLY AND PROFESSIONAL ACTIVITIES
%==============================================================================
\section*{\underline{9. SCHOLARLY AND PROFESSIONAL ACTIVITIES}}

\textit{Research or equivalent grants (indicates under COMP whether grants were obtained competitively (C) or non-competitively (NC))}

\vspace{0.5em}

\begin{longtable}{|L{1.5cm}|L{3cm}|C{1cm}|C{1.5cm}|C{2cm}|L{2cm}|L{2cm}|}
\hline
\textbf{Granting Agency} & \textbf{Subject} & \textbf{COMP} & \textbf{\$ Per Year} & \textbf{Year} & \textbf{Principal Investigator} & \textbf{Co-Investigator(s)} \\
\hline
\endfirsthead
\hline
\textbf{Granting Agency} & \textbf{Subject} & \textbf{COMP} & \textbf{\$ Per Year} & \textbf{Year} & \textbf{Principal Investigator} & \textbf{Co-Investigator(s)} \\
\hline
\endhead
NSERC & Applications of String Theory & C & 48000 & 2002--2003 & Moshe Rozali & None \\
\hline
NSERC & Applications of String Theory & C & 60000 & 2004--2006 & Moshe Rozali & None \\
\hline
NSERC & Applications of String Theory & C & 70000 & 2007--2011 & Moshe Rozali & None \\
\hline
NSERC & Development and Applications of String Theory & C & 70000 & 2012--2016 & Moshe Rozali & None \\
\hline
NSERC & Big data analysis of network metadata for optical networking systems & C & 25000 & 2015 & Moshe Rozali & Optigo Networks \\
\hline
NSERC & Applications of String Theory & C & 70000 & 2017--2022 & Moshe Rozali & None \\
\hline
NSERC & String theory, gravity and strongly interacting systems & C & 92000 & 2023--2028 & Moshe Rozali & None \\
\hline
\end{longtable}

\vspace{1em}

\textit{Invited Presentations}

\vspace{0.5em}

\begin{longtable}{|L{2.5cm}|L{3.5cm}|L{4cm}|L{3.5cm}|}
\hline
\textbf{Date} & \textbf{Location} & \textbf{Title} & \textbf{Conference} \\
\hline
\endfirsthead
\hline
\textbf{Date} & \textbf{Location} & \textbf{Title} & \textbf{Conference} \\
\hline
\endhead
April 1997 & Stanford University & Matrix Theory and U-Duality in Seven Dimensions & \\
\hline
April 1998 & ICTP, Trieste, Italy & On the (2,0) Theory in Six Dimensions & Physics of Super-Fivebranes \\
\hline
October 1998 & University of Texas at Austin & Brane Boxes, Anomalies, Bending and Tadpoles & \\
\hline
March 2000 & University of Minnesota & D-Branes in General String Backgrounds & \\
\hline
May 2000 & University of Chicago & Near Hagedorn Dynamics of NS Fivebranes & \\
\hline
October 2000 & Stanford University & Near Hagedorn Dynamics of NS Fivebranes & \\
\hline
December 2000 & Harvard University & Thermodynamics of Little String Theories & \\
\hline
January 2001 & University of Texas at Austin & Thermodynamics of Non-Gravitational String Theories & \\
\hline
February 2001 & Cornell University & High Energy Scattering in Noncommutative Field Theory & \\
\hline
February 2001 & University of Minnesota & Beyond QFT: In Search of a New Correspondence Principle (colloquium) & \\
\hline
March 2001 & University of British Columbia & Beyond QFT: In Search of a New Correspondence Principle (colloquium) & \\
\hline
March 2001 & University of British Columbia & Thermodynamics of Non-Gravitational String Theories & \\
\hline
March 2001 & UC, Irvine & Beyond QFT: In Search of a New Correspondence Principle (colloquium) & \\
\hline
June 2001 & KIAS, Seoul, Korea & Thermodynamics of Little String Theory & \\
\hline
June 2001 & KIAS, Seoul, Korea & Gauge Invariant Correlators in Noncommutative Gauge Theory & 3rd KIAS-APCTP workshop on string theory \\
\hline
October 2001 & University of Chicago & D-Branes on AdS3 & \\
\hline
February 2002 & TRIUMF & Noncommutative Field Theories -- a Survey & \\
\hline
June 2002 & April Point, Canada & PP Waves and Holography & CIAR Meeting \\
\hline
June 2002 & Neve Shalom, Israel & Plane Waves and Holography & \\
\hline
January 2003 & University of British Columbia & Introduction to String Theory (Undergraduate Physics Society) & \\
\hline
February 2003 & KIAS, Seoul, Korea & Little String Theory, Lecture Series & KIAS-APCTP 7th Winter School \\
\hline
March 2003 & University of British Columbia & String Theory for Astronomers (colloquium) & \\
\hline
December 2003 & Neve Shalom, Israel & Closed Timelike Curves and Holography in Compact Plane Waves & \\
\hline
June 2004 & Banff & Closed Strings in Misner Space & New Horizons in String Cosmology \\
\hline
September 2004 & Syracuse University & Cosmological Production of Winding Strings & \\
\hline
October 2004 & Cornell University & Cosmological Production of Winding Strings & \\
\hline
November 2004 & MIT & On Charged Black Holes in AdS Space & \\
\hline
February 2005 & Perimeter & On Charged Black Holes in AdS Space & \\
\hline
April 2005 & McGill & Singularity Resolution in Perturbative String Theory & Brane Gas Cosmology Conference \\
\hline
May 2005 & UBC & Helicity Amplitudes in Supersymmetric gauge Theories & CAP meeting \\
\hline
April 2006 & Syracuse University & Helicity amplitudes & \\
\hline
April 2006 & Perimeter Institute & Hairpin Branes and D-Branes Behind the Horizon & \\
\hline
April 2006 & Perimeter Institute & Hairpin Branes and D-Branes Behind the Horizon & Theory Canada II \\
\hline
April 2007 & Weizmann Institute & Bubbles of Nothing in AdS/CFT & \\
\hline
June 2007 & Morelia, Mexico & Background Independence in String Theory & Loops 2007 \\
\hline
February 2008 & University of Southern California & Cold Nuclear matter in Holographic QCD & \\
\hline
May 2008 & INT, Seattle & Cold Nuclear matter in Holographic QCD & From Strings to Things \\
\hline
June 2008 & University of Montreal & Cold Nuclear matter in Holographic QCD & Theory Canada 4 \\
\hline
December 2010 & Joint Theory Seminar, Neve Shalom, Israel & Holographic phase competition & \\
\hline
January 2011 & INT Seattle & Holographic phase competition & \\
\hline
February 2011 & University of Pennsylvania & Holographic phase competition & \\
\hline
February 2011 & City University of New York & Holographic phase competition & \\
\hline
March 2012 & Cambridge, UK & Amplitudes for Fivebranes & Applications of Branes in String and M-Theory \\
\hline
March 2012 & Imperial College, London & Amplitudes for Fivebranes & \\
\hline
December 2012 & M.I.T. & Inhomogeneous Holography & \\
\hline
February 2013 & Banff & Compressible Matter at a Holographic Interface & Holography and Applied String Theory \\
\hline
July 2013 & Benasque, Spain & Striped Order in AdS/CFT & Gravity -- New Perspectives from Strings and Higher Dimensions \\
\hline
August 2013 & IPMU, University of Tokyo, Japan & Inhomogeneous Holography & \\
\hline
September 2013 & Cambridge, UK & & Mathematics and Physics of the Holographic Principle \\
\hline
December 2013 & IPMU, University of Tokyo, Japan & Holographic Topological Insulators and Superconductors & \\
\hline
June 2014 & YITP, Kyoto & Holographic Edge States & Holographic Vistas on Gravity and Strings \\
\hline
December 2014 & CERN & Driven Holographic CFTs & Institute for Numerical Holography \\
\hline
April 2015 & University of Washington & Driven Holographic CFTs & \\
\hline
June 2015 & Neve Shalom, Israel & Driven Holographic CFTs & \\
\hline
March 2016 & Banff & Particle Production at Strong Coupling & Gauge/Gravity Duality and Condensed Matter Physics \\
\hline
April 2016 & Copenhagen & Particle Production at Strong Coupling & Current Themes in Holography \\
\hline
May 2016 & Technion, Israel & Evolution of Holographic Entanglement Entropy & Numerical Methods in Asymptotically AdS Spaces \\
\hline
June 2016 & Kyoto & Entanglement Propagation & Quantum Matter, Spacetime and Information \\
\hline
January 2017 & Leiden & Generalizations of the SYK model & Disorder in QFT and Holography \\
\hline
May 2017 & Florence & Quenches in the confined phase & Progress in AdS3/CFT2 holography \\
\hline
November 2017 & Stony Brook & Quenches in the confined phase & From MBL to Black Holes \\
\hline
April 2018 & UCLA & Fine Grained Chaos in AdS2 gravity & \\
\hline
May 2018 & Mainz, Germany & Fine Grained Chaos in AdS2 gravity & Modern techniques in CFT and AdS \\
\hline
August 2018 & Wurzburg, Germany & Fine Grained Chaos in AdS2 gravity & Gauge-Gravity Duality 2018 \\
\hline
August 2018 & Nordita, Sweden & Fine Grained Chaos in AdS2 gravity & Bounding transport and chaos \\
\hline
April 2019 & Neve Shalom, Israel & Effective Field Theories of Chaotic CFTs & \\
\hline
May 2020 & Spain (virtually) & ETH and disorder averaging in gravity & Holomatter \\
\hline
October 2020 & UC Davis (virtually) & ETH and disorder averaging in gravity & \\
\hline
March 2021 & Dublin (virtually) & ETH and disorder averaging in gravity & \\
\hline
October 2021 & Capetown (virtually) & Random matrix theory for complexity growth and black hole interiors & \\
\hline
November 2021 & Stony Brook (virtually) & Random matrix theory for complexity growth and black hole interiors & \\
\hline
February 2022 & Cyprus (virtually) & Random matrix theory for complexity growth and black hole interiors & \\
\hline
May 2023 & Neve Shalom, Israel & Spectral correlations in chaotic CFTs & \\
\hline
August 2023 & Mainz, Germany & Spectral correlations in chaotic CFTs & Thermalization in CFTs \\
\hline
March 2024 & Princeton, NJ & Randomness in CFTs & Random Physics \\
\hline
May 2024 & Seattle & Randomness in CFTs & PNW seminar \\
\hline
December 2024 & Caltech & Randomness in CFTs & \\
\hline
June 2025 & Pollica, Italy & Statistics of conformal field theories & Physics Sessions 2025 \\
\hline
July 2025 & Natal, Brazil & Statistics of conformal field theories & Quantum Gravity, Holography and Quantum Information \\
\hline
December 2025 & Neve Shalom, Israel & Statistics of conformal field theories & \\
\hline
December 2025 & Weizmann Institute & Matter correlators in sine-dilaton gravity & \\
\hline
December 2025 & Cologne, Germany & Statistics of conformal field theories & Field Theories of Quantum Chaos \\
\hline
\end{longtable}

\noindent(Several conference talks cancelled in 2020/1 due to covid-19)

\vspace{1em}

\textit{Conference Participation and Organization}

\noindent I have been active in co-organizing workshops, conferences and summer schools listed below. Most notable is the summer school series ``Strings, Gravity and Cosmology'' established after my arrival at UBC. In the schools we had about a dozen external speakers and 80--90 students.

\begin{itemize}[noitemsep]
    \item Pacific Northwest Seminar, March 2002, UBC, organizer.
    \item Brane-world and Supersymmetry, July 2002, UBC, organizer.
    \item Recent Developments in String Theory, March 2003, Banff, organizer.
    \item First Summer School ``Strings Gravity and Cosmology'', July 2003, chair of organizing committee.
    \item Pacific Northwest Seminar, November 2003, UBC, organizer.
    \item String Field Camp, July 2004, Banff, organizer.
    \item Second Summer School ``Strings Gravity and Cosmology'', UBC, August 2004, organizer.
    \item Third Summer School ``Strings Gravity and Cosmology'', Perimeter institute, June 2005, organizer.
    \item Fourth Summer School ``Strings Gravity and Cosmology'', UBC, August 2006, organizer.
    \item Sixth Summer school, UBC, 2008.
    \item Holographic Methods in Condensed Matter Physics, Banff, February 2016.
    \item Quantum chaos and holography, Dresden May 2022 (originally Banff March 2021).
    \item Quantum chaos and holography II, Wurzburg May 2024.
    \item Quantum chaos and holography III, Mainz June 2025.
\end{itemize}

\vspace{0.5em}

\noindent I list below participation in conferences and workshops, which are forums for informal discussions open by invitation only to a select and fairly small group of participants.

\vspace{0.5em}

\begin{longtable}{|L{3.5cm}|L{5.5cm}|L{4cm}|}
\hline
\textbf{Date} & \textbf{Name} & \textbf{Location} \\
\hline
\endfirsthead
\hline
\textbf{Date} & \textbf{Name} & \textbf{Location} \\
\hline
\endhead
April 1998 & Super-Fivebranes and Physics in 5+1 Dimensions conference & ICTP, Trieste \\
\hline
September 1998 & M-Theory and Black Holes & Aspen \\
\hline
May 1999 & Mathematics from Physics & University of Illinois \\
\hline
July 1999 & String Theory, Gauge Theory and Gravity & Amsterdam \\
\hline
August 2000 & String Dualities and Their Applications & Aspen \\
\hline
March 2001 & M-Theory & KITP, Santa-Barbara \\
\hline
August 2001 & Extreme String Theory & Aspen \\
\hline
January 2002 & Stanford-Weizmann meeting & Stanford \\
\hline
August 2003 & Time and String Theory & Aspen \\
\hline
June 2004 & New horizons in String Cosmology & Banff \\
\hline
October 2004 & Strings and QCD & KITP, Santa-Barbara \\
\hline
January 2005 & Twistor String Theory & Oxford \\
\hline
March 2005 & String Phenomenology & Perimeter \\
\hline
May 2005 & Brane Gas Cosmology & McGill \\
\hline
May 2005 & Gravitational Aspects of String Theory & Toronto \\
\hline
December 2004--May 2005 & The Geometry of String Theory & Perimeter Institute \\
\hline
August 2005 & Supercosmology & Aspen \\
\hline
January 2007 & Quantum Nature of Spacetime Singularities & Santa Barbara \\
\hline
May 2009 & Fundamental Aspects of Superstring Theory & Santa Barbara \\
\hline
June 2009 & Tom Banks/Willy Fischler 60th birthday conference & Santa Cruz \\
\hline
August 2011 & Holography and Singularities in String Theory and Quantum Gravity & Aspen, Colorado \\
\hline
November 2011 & Amplitudes 2011 conference (session chair) & University of Michigan \\
\hline
March 2012 & Applications of Branes in String and M-Theory & Cambridge, UK \\
\hline
February 2013 & Holography and Applied String Theory & Banff \\
\hline
July 2013 & Gravity -- New Perspectives from Strings and Higher Dimensions & Benasque, Spain \\
\hline
September 2013 & Mathematics and Physics of the Holographic Principle & Cambridge, UK \\
\hline
June 2014 & Holographic Vistas on Gravity and Strings & Kyoto, Japan \\
\hline
December 2014 & Institute for Numerical Holography & CERN \\
\hline
July 2015 & Gravity -- New Perspectives from Strings and Higher Dimensions & Benasque, Spain \\
\hline
August 2015 & Beyond Quasiparticles & Aspen \\
\hline
July 2017 & Gravity -- New Perspectives from Strings and Higher Dimensions & Benasque, Spain \\
\hline
June 2018 & & Florence, Italy \\
\hline
March 2019 & Many-body quantum chaos & Aspen \\
\hline
April 2019 & Machine learning and physics & Microsoft \\
\hline
July 2019 & Strings workshops & Amsterdam \\
\hline
August 2019 & Quantum information in quantum gravity 5 & Davis \\
\hline
July 2022 & Strings workshop & Amsterdam \\
\hline
August 2022 & Quantum chaos & Aspen \\
\hline
May 2023 & The physics sessions & Crete \\
\hline
July 2023 & Thermalization in CFTs & Mainz \\
\hline
March 2024 & Random Physics & Princeton \\
\hline
May 2024 & Speakable and unspeakable in quantum gravity & Saclay \\
\hline
June 2024 & Strings workshop & Amsterdam \\
\hline
June 2025 & The physics sessions & Italy \\
\hline
June 2025 & The holographic universe & Belgium \\
\hline
July 2025 & Quantum Gravity, Holography and Quantum Information & Brazil \\
\hline
\end{longtable}

\noindent(everything in 2020 and 2021 cancelled, I did not list a few online workshops)

\vspace{1em}

%==============================================================================
% SECTION 10: SERVICE TO THE UNIVERSITY
%==============================================================================
\section*{\underline{10. SERVICE TO THE UNIVERSITY}}

\subsection*{(a) \textit{Memberships on committees, including offices held and dates}}

\begin{itemize}[noitemsep]
    \item Hiring Committee, Particle theory search, 2004.
    \item Graduate Awards Chair 2005--2008.
    \item Colloquium Committee, 2008--present, chair 2009.
    \item Department newsletter editor, 2010--present.
    \item Committee of Initial Appointments, 2008--present. Chair, since 2023.
    \item Committee of Promotion and Tenure, 2008--present.
    \item Graduate Admissions, 2012--present
\end{itemize}

\subsection*{(b) \textit{Other service, including dates}}

\begin{itemize}[noitemsep]
    \item Member of Organizing committee of the department retreat, April 2002.
    \item Organizer, string theory seminar 2001--2002.
    \item Organizer, theory seminar, 2003.
\end{itemize}

\vspace{1em}

%==============================================================================
% SECTION 11: SERVICE TO THE COMMUNITY
%==============================================================================
\section*{\underline{11. SERVICE TO THE COMMUNITY}}

\subsection*{(a) \textit{Memberships on scholarly societies, including offices held and dates:}}

\noindent Member of The Foundational Questions Institute (FQXI), starting spring 2011:

\noindent FQXI is a private institute dedicated to exploring issues in the forefront of Physics. Membership, which is extended by invitation only, is relatively small, consisting of a few dozen members. Members are invited to FQXI conferences and workshops and are entitled to apply to ``mini-grants'' -- grant competitions open to members only.

\noindent Member of Compute Canada and Westgrid user.

\noindent Member of American Physical Society, Canadian Association of Physics, Institute of Particle Physics, Institute of Physics (UK).

\subsection*{(b) \textit{Memberships on scholarly committees, including offices held and dates}}

\noindent Member of NSERC's ad hoc committee for long range planning for theoretical subatomic physics in Canada, 2005--2006.

\noindent Member of IPP postdoc selection committee, 2007.

\noindent Member of NSERC's grant selection committee 19 (subatomic physics), 2008--2011, ad hoc member 2014, 2016.

\noindent The subatomic section is unique in NSERC is assigning grants (including grant amounts) from a fixed envelope. The membership rotates every 3 years and includes, in addition to making grant decisions, membership in standing review committees for large facilities (e.g. TRIUMF or ATLAS Canada), and participation in departmental visits across Canada.

\subsection*{(c) \textit{Reviewer (journal, agency, etc. including dates)}}

\noindent Referee for various journals including for example: Physics Review, Nuclear Physics B, Journal of High Energy Physics, Journal of Canadian Physics, Classical and Quantum Gravity, and others.

\noindent Referee for NSERC discovery grant proposals.

\noindent Referee for DOE grant proposals (early career, theoretical condensed matter sections).

\subsection*{(d) \textit{External examiner (indicate universities and dates)}}

\begin{itemize}[noitemsep]
    \item David Winters, McGill University, 2003.
    \item Aaron James Berndsen, McGill, 2007.
\end{itemize}

\vspace{1em}

%==============================================================================
% SECTION 12: AWARDS AND DISTINCTIONS
%==============================================================================
\section*{\underline{12. AWARDS AND DISTINCTIONS}}

\subsection*{(a) \textit{Awards for Teaching (indicate name of award, awarding organizations, date)}}

\subsection*{(b) \textit{Awards for Scholarship (indicate name of award, awarding organizations, date)}}

\noindent First prize essay in the gravity foundation essay contest (with V. Balasubramanian and D. Marolf), April 2006.

\noindent The gravity research foundation maintains an annual contest for research essays on topics related to gravitational physics. The essays are received from all over the world and are judged by a committee of professional physicists. Past winners include Steven Hawking, Roger Penrose, Paul Steinhardt, Lawrence Krauss, Ilya Prigogine, A.M. Polyakov, and other prominent physicists.

\vspace{0.5em}

\noindent Third prize essay in the FQXI essay competition, Spring 2011.

\noindent The Foundational Questions Institute maintains occasional contests for essays on the foundations of physics and related topics. The 2011 competition was titled ``Is Reality Digital and Analogue'' and drew a few hundred essays from around the world. The third prize is 2000 USD.

\newpage

%==============================================================================
% PUBLICATIONS RECORD
%==============================================================================
\cvheader{THE UNIVERSITY OF BRITISH COLUMBIA}
\cvsubheader{Publications Record}

\vspace{1em}

\noindent\textbf{SURNAME:} Rozali \hfill \textbf{FIRST NAME:} Moshe

\vspace{0.5em}

\noindent\textbf{Initials:} \hfill \textbf{Date:} 3 April 2025

\vspace{1em}

\noindent\textbf{Note:} The authorship convention in high-energy theoretical physics does not include any hierarchy between different authors. The concepts of first or last author do not exist. The names in the publications below are listed alphabetically, in accordance with the convention in my field of study. I have highlighted (in boldface) the HQP who were training at UBC at the time.

\vspace{1em}

%------------------------------------------------------------------------------
% REFEREED PUBLICATIONS
%------------------------------------------------------------------------------
\section*{\underline{1. REFEREED PUBLICATIONS}}

\begin{enumerate}[leftmargin=*, label=\arabic*.]

\item Moshe Rozali, ``Non-Perturbative Decoupling of Heavy Fermions'', Published in Physics Letters B345, 507 (1995).

\item Willy Fischler, Sonia Paban, Moshe Rozali, ``Collective Coordinates in String Theory'', Published in Physics Letters B352, 298 (1995).

\item Willy Fischler, Sonia Paban, Moshe Rozali, ``Collective Coordinates for D-Branes'', Published in Physics Letters B381, 62 (1996).

\item Richard Corrado, David Berenstein, Willy Fischler, Sonia Paban, Moshe Rozali, ``Virtual D-Branes'', Published in Physics Letters B384, 93 (1996).

\item Moshe Rozali, ``Matrix Theory and U-Duality in Seven Dimensions'', Published in Physics Letters B400, 260 (1997).

\item Micha Berkooz, Moshe Rozali, Nathan Seiberg, ``Matrix Description of M Theory on T\^{}4 and T\^{}5'', Published in Physics Letters B408, 105 (1997).

\item Micha Berkooz, Moshe Rozali, ``String Dualities from Matrix Theory'', Published in Nuclear Physics B516, 229 (1998).

\item Robert Leigh, Moshe Rozali, ``A Note on Six Dimensional Gauge Theories'', Published in Physics Letters B433, 43 (1998).

\item Robert Leigh, Moshe Rozali, ``The Large N Limit of the (2,0) Superconformal Field Theory'', Published in Physics Letters B431, 311 (1998).

\item Robert Leigh, Moshe Rozali, ``Brane Boxes, Anomalies, Bending and Tadpoles'', Published in Physical Review D59, 026004 (1999).

\item Arvind Rajaraman, Moshe Rozali, ``On the Quantization of the GS String on AdS\_5 times S\^{}5'', Published in Physics Letters B468, 58 (1999).

\item Ilka Brunner, Arvind Rajaraman, Moshe Rozali, ``D-Branes on Asymmetric Orbifolds'', Published in Nuclear Physics B558, 205 (1999).

\item Arvind Rajaraman, Moshe Rozali, ``D-Branes in Linear Dilaton Backgrounds'', Published in Journal of High Energy Physics 9912, 005 (1999).

\item Moshe Rozali, ``Hypermultiplet Moduli Space and Three Dimensional Gauge Theories'', Published in Journal of High Energy Physics, 9912, 013 (1999).

\item Arvind Rajaraman, Moshe Rozali, ``Noncommutative Gauge Theory, Divergences and Closed Strings'', Published in Journal of High Energy Physics 0004, 033 (2000).

\item Micha Berkooz, Moshe Rozali, ``Near Hagedorn Dynamics of NS Fivebranes, or A New Universality Class of Coiled Strings'', Published in Journal of High Energy Physics 0005, 040 (2000).

\item Mark van Raamsdonk, Moshe Rozali, ``Gauge Invariant Correlators in Noncommutative Gauge Theory'', Published in Nuclear Physics B608, 103 (2001).

\item Moshe Rozali, ``High Energy Scattering in Noncommutative Gauge Theory'', Published in Journal of Korean Physical Society 39, s584 (2001).

\item Arvind Rajaraman, Moshe Rozali, ``Boundary States for D-branes on AdS3'', Published in Physics Review D66, 026006 (2002).

\item Robert Leigh, Kazumi Okuyama, Moshe Rozali, ``PP-Waves and Holography'', Published in Physical Review D66, 046004 (2002).

\item \textbf{Philip DeBoer}, Moshe Rozali, ``Thermal Correlators in Little String Theory'', Published in Physical Review D67, 086009 (2003).

\item Joel Geidt, Erich Poppitz, Moshe Rozali, ``Deconstruction, Lattice Supersymmetry, Anomalies and Branes'', Published in Journal of High Energy Physics 0303, 035 (2003).

\item \textbf{Dominic Brecher}, \textbf{Philip DeBoer}, David Page, Moshe Rozali, ``Closed Timelike Curves and Holography in Compact Plane Waves'', Published in Journal of High Energy Physics 0310, 031 (2003).

\item Micha Berkooz, Boris Pioline, Moshe Rozali, ``Closed Strings in Misner Space'', Published in Journal of Cosmology and Astrophysics 0408, 004 (2004).

\item \textbf{Dominic Brecher}, \textbf{Jianyang He}, Moshe Rozali, ``On Charged Black Holes in Anti-de-Sitter Space'', Published in Journal of High Energy Physics 0504, 004 (2005).

\item \textbf{Callum Quigley}, Moshe Rozali, ``One-Loop MHV Amplitudes in Supersymmetric Gauge Theories'', Published in Journal of High Energy Physics 0501, 053 (2005).

\item \textbf{Callum Quigley}, Moshe Rozali, ``Recursion Relations, Helicity Amplitudes and Dimensional Regularization'', Published in Journal of High Energy Physics 0603, 004 (2006).

\item \textbf{Kazumi Okuyama}, Moshe Rozali, ``Hairpin Branes and D-Branes Behind the Horizon'', Published in Journal of High Energy Physics 0603, 071 (2006).

\item Vijay Balasubramanian, Don Marolf, Moshe Rozali, ``Information Recovery from Black Holes'', First prize in the Gravity Research Foundation essay competition, Published in General Relativity and Gravitation 38, 1529 (2006), Reprinted in International Journal of Modern Physics D15, 228 (2006).

\item Moshe Rozali, ``D-Branes Behind The Horizon'', Published in the Canadian Journal of Physics 85, 619 (2007).

\item \textbf{Jianyang He}, Moshe Rozali, ``On Bubbles of Nothing in AdS/CFT'', Published in Journal of High Energy Physics 0709, 089 (2007).

\item Moshe Rozali, \textbf{Brian Shieh}, Mark Van Raamsdonk and \textbf{Jackson Wu}, ``Cold Nuclear Matter in Holographic QCD'', Published in Journal of High Energy Physics 0801, 053 (2008).

\item \textbf{Lionel Brits}, Moshe Rozali, ``Holography and Fermions at Finite Chemical Potential'', Published in Canadian Journal of Physics, 87, 271 (2009).

\item Moshe Rozali, ``Comments on Background Independence and gauge Redundancies'', Published in Advanced Science Letters 2, 244 (2009).

\item \textbf{Pallab Basu}, \textbf{Jianyang He}, \textbf{Anindya Mukherjee}, Moshe Rozali, \textbf{Hsein-Hang Shieh}, ``Holographic Phase Competition'', Published in JHEP 1010, 092 (2010).

\item \textbf{Pallab Basu}, \textbf{Jianyang He}, \textbf{Anindya Mukherjee}, Moshe Rozali, \textbf{Hsein-Hang Shieh}, ``Comments on Non-Fermi Liquids in the presence of a Condensate'', Published in the Canadian Journal of Physics 89, 231 (2011).

\item \textbf{Pallab Basu}, \textbf{Fernando Nogueira}, Moshe Rozali, \textbf{Jared Stang} and Mark van Raamsdonk, ``Towards A Holographic Models of Color Superconductivity'', Published in New Journal of Physics 13, 055001 (2011).

\item \textbf{Bartek Czech}, \textbf{Klaus Larjo} and Moshe Rozali, ``Black Holes as Rubik's Cubes'', Published in Journal of High Energy Physics 1108, 143 (2011).

\item \textbf{Bartek Czech}, YuTin Huang and Moshe Rozali, ``Chiral 3 point Interactions in 5 and 6 Dimensions'', Published in Journal of High Energy Physics, 1210, 143 (2012).

\item Moshe Rozali, \textbf{Darren Smyth} and \textbf{Evgeny Sorkin}, ``Holographic Higgs Phases'', Published in Journal of High Energy Physics 1208, 118 (2012).

\item Moshe Rozali, ``Compressible Matter at an Holographic Interface'', Published in Physical Review Letters 109 (2012), 231601.

\item Moshe Rozali, \textbf{Darren Smyth}, \textbf{Evgeny Sorkin} and \textbf{Jared B. Stang}, ``Holographic Stripes'', Published in Physical Review Letters 110 (2013), 201603.

\item \textbf{Darren Smyth}, \textbf{Evgeny Sorkin}, \textbf{Jared B. Stang} and Moshe Rozali, ``Striped Order in AdS/CFT'', Published in Physical Review D. 87, 126007 (2013).

\item Moshe Rozali, \textbf{Jared B. Stang} and Mark van Raamsdonk, ``Holographic Baryons from Oblate Instantons'', Published in Journal of High Energy Physics 1402, 044 (2014).

\item Moshe Rozali and \textbf{Alexandre Vincart-Emard}, ``Chiral Edge Currents in a Holographic Josephson Junction'', Published in Journal of High Energy Physics 1401, 003 (2014).

\item Tomas Andrade, Sebastian Fischetti, Don Marolf, Simon F. Ross and Moshe Rozali, ``Entanglement and Correlations Near Extremality: CFTs Dual to Reissner-Nordstrom AdS5'', Published in Journal of High Energy Physics 1404, 023 (2014).

\item Moshe Rozali, \textbf{Darren Smyth}, ``Fermi Liquids from D-Branes'', Published in Journal of High Energy Physics 1405, 129 (2014).

\item Mukund Rangamani, Moshe Rozali, \textbf{Anson Wong}, ``Driven Holographic CFTs'', Published in Journal of High Energy Physics, 1504, 093 (2015).

\item Mukund Rangamani, Moshe Rozali, Mark van Raamsdonk, ``Cosmological Particle Production at Strong Coupling'', Published in Journal of High Energy Physics 1509, 213 (2015).

\item Mukund Rangamani, Moshe Rozali, \textbf{Darren Smyth}, ``Spatial Modulation and Conductivities in Effective Holographic Theories'', Published in Journal of High Energy Physics 1507, 024 (2015).

\item Mukund Rangamani, Moshe Rozali, \textbf{Alexandre Vincart-Emard}, ``Dynamics of Holographic Entanglement Entropy following a Local Quench'', Published in Journal of High Energy Physics 1604, 069 (2016).

\item Moshe Rozali and \textbf{Alexandre Vincart-Emard}, ``On Brane Instabilities in the large D Limit'', Published in Journal of High Energy Physics 1608, 166 (2016).

\item Micha Berkooz, Prithvi Narayan, Moshe Rozali and Joan Simon, ``Higher Dimensional Generalizations of the SYK Model'', Published in Journal of High Energy Physics 1701, 138 (2017).

\item Micha Berkooz, Prithvi Narayan, Moshe Rozali and Joan Simon, ``Comments on the Random Thirring Model'', Published in Journal of High Energy Physics 1709, 059 (2017).

\item Moshe Rozali and \textbf{Alexandre Vincart-Emard}, ``Comments on Entanglement Propagation'', Published in Journal of High Energy Physics 1706, 044 (2017).

\item Rob Myers, Moshe Rozali and \textbf{Benson Way}, ``Holographic Quenches in a Confined Phase'', Published in Journal of Physics A50, no. 49, 494002 (2017). (Issue dedicated to John Cardy)

\item Moshe Rozali, Evyatar Sabag and Amos Yarom, ``Holographic Turbulence in a Large Number of Dimensions'', Published in Journal of High Energy Physics 1804 (2018) 065.

\item Moshe Rozali, \textbf{Felix Haehl}, ``Fine Grained Chaos in AdS2 Gravity'', Published in Physical Review Letters 120 (2018) no.12, 121601.

\item Moshe Rozali and \textbf{Benson Way}, ``Gravitating Scalars and Critical Collapse in the Large D Limit'', Published in Journal of High Energy Physics 1811, 106 (2018).

\item Marcel Franz, Moshe Rozali, ``Mimicking Black Hole Event Horizons in Atomic and Solid-State System'', Published in Nature Review Materials 4, 491 (2018).

\item \textbf{Felix Haehl}, Moshe Rozali, ``Effective Field Theory of Chaotic CFTs'', Published in Journal of High Energy Physics 1810, 118 (2018).

\item \textbf{Sean Cooper}, Moshe Rozali, Brian Swingle, Mark van Raamsdonk, \textbf{Chris Waddell}, \textbf{David Wakeham}, ``Black Hole Microstate Cosmology'', Published in JHEP 07 (2019) 065.

\item \textbf{Felix Haehl}, \textbf{Wyatt Reeves} and Moshe Rozali, ``Reparametrization modes, shadow operators, and quantum chaos in higher-dimensional CFTs'', Published in JHEP 11 (2019) 102.

\item Moshe Rozali, \textbf{Jamie Sully}, Mark van Raamsdonk, \textbf{Chris Waddell}, \textbf{David Wakeham}, ``Information radiation in BCFT models of black holes'', Published in JHEP 05 (2020) 004.

\item \textbf{Sean Cooper}, \textbf{Dominik Neuenfeld}, Moshe Rozali, \textbf{David Wakeham}, ``Brane dynamics from the first law of entanglement'', Published in JHEP 03 (2020) 023.

\item \textbf{Jason Pollack}, Moshe Rozali, \textbf{Jamie Sully} and \textbf{David Wakeham}, ``Eigenstate Thermalization and Disorder Averaging in Gravity'', Published in PRL 125 (2020) 2, 021601.

\item \textbf{Arjun Kar}, \textbf{Lampros Lamprou}, Moshe Rozali and \textbf{James Sully}, ``Random matrix theory for complexity growth and black hole interiors'', Published in JHEP 01 (2022) 016.

\item \textbf{Wyatt Reeves}, Moshe Rozali, \textbf{Petar Simidzjiya}, \textbf{James Sully}, \textbf{Christopher Waddell}, ``Looking for (and not finding) a bulk brane'', Published in JHEP 12 (2021) 002.

\item \textbf{Felix Haehl}, \textbf{Charles Marteau}, \textbf{Wyatt Reeves}, Moshe Rozali, ``Symmetries and spectral statistics in chaotic conformal field theories'', JHEP 07 (2023) 196.

\item \textbf{Felix Haehl}, \textbf{Wyatt Reeves}, Moshe Rozali, ``Symmetries and spectral statistics in chaotic conformal field theories II: Maass cusp forms and arithmetic chaos'', JHEP 12 (2023) 161.

\item \textbf{Felix Haehl}, \textbf{Wyatt Reeves}, Moshe Rozali, ``Euclidean wormholes in two-dimensional conformal field theories from quantum chaos and number theory'', Phys.Rev.D 108 (2023) 10.

\item \textbf{Chuanxin Cui}, Moshe Rozali, ``Comments on firewalls in JT gravity with matter'', JHEP 03 (2025) 104.

\item Jan Boruch, Gabriele Di Ubaldo, Felix M. Haehl, Eric Perlmutter, Moshe Rozali, ``Modular-invariant random matrix theory and AdS3 wormholes'', Published in Phys.Rev.Lett. 135 (2025) 12, 121602.

\item Alexander Altland, \textbf{Jeremy van der Heijden}, Tobias Micklitz, Moshe Rozali, Joaquim Telles de Miranda, ``Universality Class of the First Levels in Low-Dimensional Gravity'', Published in Phys.Rev.Lett. 135 (2025) 12, 121601.

\item \textbf{Chuanxin Cui}, Moshe Rozali, ``Splitting and gluing in sine-dilaton gravity: matter correlators and the wormhole Hilbert space'', Submitted to JHEP.

\item Gabriele Di Ubaldo, Altay Etkin, Felix M. Haehl, Moshe Rozali, ``Mind the crosscap: N-scaling in non-orientable gravity and time-reversal-invariant system'', Submitted to JHEP.

\end{enumerate}

\vspace{1em}

%------------------------------------------------------------------------------
% WORK SUBMITTED OR IN PREPARATION
%------------------------------------------------------------------------------
\section*{\underline{2. WORK SUBMITTED OR IN PREPARATION}}

\vspace{1em}

%------------------------------------------------------------------------------
% NON-REFEREED PUBLICATIONS
%------------------------------------------------------------------------------
\section*{\underline{3. NON-REFEREED PUBLICATIONS}}

\noindent Moshe Rozali, ``Why 10 or 11 Dimensions?'' in ``Ask the Experts'', Scientific American, February 2006.

\noindent Moshe Rozali, ``Continuous Spacetimes from Discrete Holographic Models'', winner of 3rd prize in the FQXI essay competition ``Is Reality Digital or Analog?'' January 2011.

\end{document}
